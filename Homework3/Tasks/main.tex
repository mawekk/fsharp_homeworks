\documentclass{article}
\usepackage{amsmath}
\usepackage[T2A]{fontenc}

\begin{document}

\section*{\LARGE Task 1}
\LARGE$$(\lambda a.\ (\lambda b. b\ b)\ (\lambda b. b\ b)\ b)\ ((\lambda c. (c\ b))\ (\lambda a. a)) \rightarrow_\beta$$
$$((\lambda b. b\ b)\ (\lambda b. b\ b))\ ((\lambda c. (c\ b))\ (\lambda a. a)) \rightarrow_\beta$$
$$(\omega\ \omega)\ ((\lambda c. (c\ b))\ (\lambda a. a)) \equiv$$
$$\Omega\ ((\lambda c. (c\ b))\ (\lambda a. a))$$

Используя нормальную стратегию, получаем расходящийся комбинатор, не имеющий нормальной формы. 
Значит, у этого \(\lambda\)-терма нет нормальной формы.

\section*{\LARGE Task 2}
\LARGE$$S\ K\ K\ \equiv (\lambda x\ y\ z.\ x\ z\ (y\ z))\ (\lambda x\ y.\ x)\ (\lambda x\ y.\ x) \rightarrow_\beta$$
$$(\lambda y\ z.\ (\lambda x\ y.\ x)\ z\ (y\ z))\ (\lambda x\ y.\ x) \rightarrow_\beta$$
$$(\lambda z.\ (\lambda x\ y.\ x)\ z\ ((\lambda x\ y.\ x)\ z)) \rightarrow_\beta$$
$$(\lambda z.\ (\lambda y.\ z)\ ((\lambda x\ y.\ x)\ z)) \rightarrow_\beta$$
$$(\lambda z.\ (\lambda y.\ z)\ (\lambda y.\ z) \rightarrow_\beta$$
$$\lambda z.\ z \equiv I\ $$

\end{document}
